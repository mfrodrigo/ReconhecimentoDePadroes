\documentclass[12pt]{article}
\usepackage[utf8]{inputenc}
%\usepackage[portuguese]{babel}
\usepackage{amsmath,amsfonts,amssymb}
\usepackage{graphicx}
\usepackage{makeidx}
\usepackage{graphicx}
\usepackage{lmodern}
\usepackage{multicol}
\usepackage{booktabs}
\usepackage{fancyhdr}
\usepackage{hyperref}
\usepackage[usenames]{color}


\usepackage{Sweave}
\begin{document}
\Sconcordance{concordance:LSSVM.tex:LSSVM.Rnw:%
1 15 1 1 0 62 1 1 10 9 0 1 2 4 1 1 13 1 2 6 0 1 2 2 1 1 25 4 1 1 2 1 0 %
1 1 1 15 17 0 1 2 1 1 1 2 12 0 1 2 12 0 1 2 16 1}

\pagestyle{fancy}
\fancyhf{}
\renewcommand{\headrulewidth}{0.4pt}
\fancyfoot[C]{\thepage}
\renewcommand{\footrulewidth}{0.4pt}
\fancyfoot[C]{\thepage}
\title{\LARGE \bf
 Exercício 6  -  Aplicação LSSVM}
\author{ Rodrigo Machado Fonseca - 2017002253}
\thispagestyle{fancy}
\fancyhead[C]{Introdução ao Reconhecimento de Padrões - UFMG \\ Belo Horizonte - \today}
\maketitle
\thispagestyle{fancy}

%%%%%%%%%%%%%%%%%%%%%%%%%%%%%%%%%%%%%%%%%%%%%%%%%%%%%%%%%%%%%%%%%%%%%%%%%%%%%%%%%%%%%%%%%
\section{Introdução}

  \par Neste exercício utilizaremos o \textit{Least Square Support Vector Machine} (LSSVM) para resolver o problema de classificação de tipos de vidros do banco de dados Glass (base nativa da Linguagem), a partir de suas características químicas.

\section{Least Square SVM}

  \par \textit{Least Square SVM} (LSSVM) são versões de mínimos quadrados de máquinas de vetores de suporte (SVM). Nesta versão, encontra-se a solução resolvendo um conjunto de equações lineares em vez de um problema de programação quadrática convexa para SVMs clássicos. Algoritmos eficientes e escaláveis, como aqueles baseados em gradiente, podem ser aplicados para resolver LSSVM.
  
  \par A diferença entre o SVM e o LSSVM baseia-se, principalmente, na definição do erro. Neste caso definimos o erro como uma igualdade, conforme a equação a seguir:
  
\begin{equation}
min_{w, v,e} J_p(w,v,e)=0.5w^{T}w+\gamma 0.5\sum_{i=1}^{N}e_i^2
\end{equation}

sujeito a 

\begin{equation}
y_i[w^T\varphi(X_i)+b]=1-e_i, \ \ \ \ \ \ i=1,......,N
\end{equation}

  \par O $\gamma$ é o parâmetro de regularização e estar relacionado as variáveis de folga.
  
  \par O problema está sujeito as seguintes restrições:
  
\begin{equation}
  w = \sum_{i}^{N} \alpha_iy_i\varphi(x_i)
\end{equation}

\begin{equation}
  \sum_{i}^{N} \alpha_iy_i = 0
\end{equation}

\begin{equation}
  \alpha_i = \gamma e_i \ \ \forall_i
\end{equation}

\begin{equation}
y_i[w^T\varphi(X_i)+b]-1 + e_i = 0 \ \ \forall_i
\end{equation}

  \par A condição 3 possibilita que a LSSVM não resulte, necessariamente, em valores de $\alpha$ nulos, como em SVM. O parâmetro $\gamma$ é uma proporcionalidade, não limitando assim o valor superior de $\alpha_i$.
  

\section{Metodologia}

  \par A priori, será carregada a base de dados Glass do R. A seguir pode-se ver uma pequena amostra da base de dados:
\begin{Schunk}
\begin{Soutput}
       RI    Na   Mg   Al    Si    K   Ca Ba Fe Type
1 1.52101 13.64 4.49 1.10 71.78 0.06 8.75  0  0    1
2 1.51761 13.89 3.60 1.36 72.73 0.48 7.83  0  0    1
3 1.51618 13.53 3.55 1.54 72.99 0.39 7.78  0  0    1
4 1.51766 13.21 3.69 1.29 72.61 0.57 8.22  0  0    1
5 1.51742 13.27 3.62 1.24 73.08 0.55 8.07  0  0    1
\end{Soutput}
\end{Schunk}

  \par Em sequência, iremos utilizar a normalização para que todas variáveis possam ter o mesmo peso no modelo, com a seguinte equação:
\begin{equation}
x'_{i} = \frac{x_i - min(x)}{max(x) - min(x)}
\end{equation}
  \par Neste problema, as classes estão desbalanceadas, como pode ser visto a seguir, onde cada coluna representa o \textit{label} e a linha representa a quantidade de dados de cada classe:
\begin{Schunk}
\begin{Soutput}
y
 1  2  3  5  6  7 
70 76 17 13  9 29 
\end{Soutput}
\end{Schunk}

 \par Por fim, iremos separar 70\%  do conjunto para treino e 30\% para teste. 


\section{Resultados}
  
  \par Neste experimento iremos para cada conjunto de parâmetros "kpar" e "tau"($\gamma$) faremos ele 10 vezes o experimento e por fim analisaremos a média e o desvio padrão.
  
\begin{Schunk}
\begin{Sinput}
> set.seed(1)
> accuracy_final <- matrix(nrow=10, ncol=9)
> for(i in 1:10){
+   index <- createDataPartition(y,
+                                p = 0.7, 
+                                list = FALSE)
+   x_train <- x[index,]
+   y_train <- y[index]
+   x_test <- x[-index,]
+   y_test <- y[-index]
+   r <- c(0.1, 0.1, 0.1, 0.2, 0.3, 0.4, 0.1, 0.1, 0.1)
+   tau <- c(0.3, 0.45, 0.5, 0.5, 0.5, 0.9, 0.2, 0.1, 0.8)
+   results <- training_lssvm(x_train, y_train, x_test, y_test, r, tau)
+   svm <- results[[1]]
+   accuracy_list <- results[[2]]
+   accuracy_final[i,] <- as.matrix(accuracy_list, nrow=1)
+ }
\end{Sinput}
\end{Schunk}

  \par Nas tabelas a seguir estão representadas as médias e os desvios padrão para cada experimento.
\begin{Schunk}
\begin{Soutput}
  colMeans.accuracy_final.
1                0.5730159
2                0.5809524
3                0.5777778
4                0.5666667
5                0.5761905
6                0.5492063
7                0.5777778
8                0.5825397
9                0.5730159
\end{Soutput}
\begin{Soutput}
          sd
1 0.03214041
2 0.03604103
3 0.03604103
4 0.02487335
5 0.03265883
6 0.02504161
7 0.02718568
8 0.02804732
9 0.03774827
\end{Soutput}
\end{Schunk}


\section{Discussão}

  \par Neste exercício, assim como no anterior, os parâmetros da função utilizada para treinar foram ajustadas, a fim de obter o melhor resultado possível. Após o problema chegar a $58\%$ de acurácia, ele estagnou e não foi possível aumentar a acurácia. Tentamos deixar o parâmetro "kpar" automático, porém isso não se motrou mais vantajoso. 
  
  \par É possível notar que a acurácia média obtida tem valor menor do que aquela obtida no exercício anterior, que tratava do mesmo problema, porém utilizando o classificador SVM. Este resultado nos leva a crer que as SVMs, apesar de apresentarem carga computacional maior, parecem apresentar desempenho um pouco melhor. O único ponto que devemos tomar cuidado é o fato que o as parâmetros foram definidos de forma arbitrária e sem nenhum rigor matemático.
  
  \par Logo, com o experimento foi possível compreender melhor o comportamento do algoritmo LSSVM, além de que conseguimos construir e validar um classificador LSSVM. É possível afirmar que a prática foi bem executada e o classificador construído apresentou um resultado razoável.
  
%%%%%%%%%%%%%%%%%%%%%%%%%%%%%%%%%%%%%%%%%%%%%%%%%%%%%%%%%%%%%%%%%%%%%%%%%%%%%%%%%%%%%%%%%

%%%%%%%%%%%%%%%%%%%%%%%%%%%%%%%%%%%%%%%%%%%%%%%%%%%%%%%%%%%%%%%%%%%%%%%%%%%%%%%%%%%%%%%%%



\end{document}
