\documentclass[12pt]{article}
\usepackage[utf8]{inputenc}
%\usepackage[portuguese]{babel}
\usepackage{amsmath,amsfonts,amssymb}
\usepackage{graphicx}
\usepackage{makeidx}
\usepackage{graphicx}
\usepackage{lmodern}
\usepackage{multicol}
\usepackage{booktabs}
\usepackage{fancyhdr}
\usepackage{hyperref}
\usepackage[usenames]{color}


\usepackage{Sweave}
\begin{document}
\Sconcordance{concordance:knn.tex:knn.Rnw:%
1 15 1 1 0 43 1 1 2 1 0 1 5 4 0 1 25 26 0 1 2 8 1 1 19 18 0 2 1 1 2 1 0 %
1 1 3 0 1 2 1 1 1 14 16 0 1 2 3 1 1 8 10 0 1 2 2 1 1 3 1 2 3 1 1 4 2 1 %
1 3 1 2 5 1 1 3 1 2 5 1 1 3 1 2 6 1 1 8 2 1 1 3 1 2 3 1 1 4 2 1 1 3 1 2 %
5 1 1 3 1 2 5 1 1 3 1 2 4 1 1 8 2 1 1 3 1 2 3 1 1 4 2 1 1 3 1 2 5 1 1 3 %
1 2 5 1 1 3 1 2 23 1}

\pagestyle{fancy}
\fancyhf{}
\renewcommand{\headrulewidth}{0.4pt}
\fancyfoot[C]{\thepage}
\renewcommand{\footrulewidth}{0.4pt}
\fancyfoot[C]{\thepage}
\title{\LARGE \bf
 Exercício 3 -  K-Nearest Neighbors}
\author{ Rodrigo Machado Fonseca - 2017002253}
\thispagestyle{fancy}
\fancyhead[C]{Introdução ao Reconhecimento de Padrões - UFMG \\ Belo Horizonte - \today}
\maketitle
\thispagestyle{fancy}

%%%%%%%%%%%%%%%%%%%%%%%%%%%%%%%%%%%%%%%%%%%%%%%%%%%%%%%%%%%%%%%%%%%%%%%%%%%%%%%%%%%%%%%%%
\section{Introdução}

  \par Neste trabalho iremos implementar o algoritmo \textit{k-Nearest Neighbors} na secão \ref{knn}. Em seguida, iremos utilizá-lo para classificar um conjunto de amostras. 
  
\section{K-Nearest Neighbors}
  \label{knn}
  \par O classificador \textit{k-Nearest Neighbors} é um classificador que utiliza métricas de distâncias para classificar novas amostras. Para entendermos como funcionar vamos analisar o conjunto de passos:

  \begin{itemize}
    \item Recebe uma amostra e calcula a distância para todas as amostras que já estão rótuladas. 
    \item Ordena em ordem crescente de acordo com a distância calculada.
    \item Escolhe as k primeiras amostras.
    \item Conta o número de rótulos dentro do grupo escolhido.
    \item Ordena em ordem decrescente. 
    \item Atribui como rótulo da nova amostra o primeiro rótulo da lista. 
  \end{itemize}

  \par Para aplicar o método basta termos um conjunto de amostras que já estão rótuladas e a partir do momento que recebemos um novo número de amostras, podemos implementar a sequência de passos acima que teremos uma classificação para nova amostras. 

%%%%%%%%%%%%%%%%%%%%%%%%%%%%%%%%%%%%%%%%%%%%%%%%%%%%%%%%%%%%%%%%%%%%%%%%%%%%%%%%%%%%%%%%%

%%%%%%%%%%%%%%%%%%%%%%%%%%%%%%%%%%%%%%%%%%%%%%%%%%%%%%%%%%%%%%%%%%%%%%%%%%%%%%%%%%%%%%%%%


\begin{thebibliography}{99}
		\bibitem{c1}\label{BreastCancer}
\end{thebibliography}	


\end{document}
