\documentclass[12pt]{article}
\usepackage[utf8]{inputenc}
%\usepackage[portuguese]{babel}
\usepackage{amsmath,amsfonts,amssymb}
\usepackage{graphicx}
\usepackage{makeidx}
\usepackage{graphicx}
\usepackage{lmodern}
\usepackage{multicol}
\usepackage{booktabs}
\usepackage{fancyhdr}
\usepackage{hyperref}
\usepackage[usenames]{color}


\usepackage{Sweave}
\begin{document}
\Sconcordance{concordance:SVM_Application.tex:SVM_Application.Rnw:%
1 15 1 1 0 23 1 1 9 9 0 1 2 4 1 1 13 1 2 6 0 1 2 1 1 1 37 39 0 1 2 1 23 %
4 1 1 2 1 0 1 1 1 13 15 0 1 2 1 1 1 2 12 0 1 2 12 0 1 2 15 1}

\pagestyle{fancy}
\fancyhf{}
\renewcommand{\headrulewidth}{0.4pt}
\fancyfoot[C]{\thepage}
\renewcommand{\footrulewidth}{0.4pt}
\fancyfoot[C]{\thepage}
\title{\LARGE \bf
 Exercício 5  -  Aplicação SVM}
\author{ Rodrigo Machado Fonseca - 2017002253}
\thispagestyle{fancy}
\fancyhead[C]{Introdução ao Reconhecimento de Padrões - UFMG \\ Belo Horizonte - \today}
\maketitle
\thispagestyle{fancy}

%%%%%%%%%%%%%%%%%%%%%%%%%%%%%%%%%%%%%%%%%%%%%%%%%%%%%%%%%%%%%%%%%%%%%%%%%%%%%%%%%%%%%%%%%
\section{Introdução}

  \par Neste exercício utilizaremos o \textit{Support Vector Machine} (SVM) para resolver o problema de classificação de tipos de vidros do banco de dados Glass (base nativa da Linguagem) a partir de suas características químicas.
  
\section{Metodologia}

  \par A priori, será carregada a base de dados Glass do R. A seguir pode-se ver um pequena amostra da base de dados:
\begin{Schunk}
\begin{Soutput}
       RI    Na   Mg   Al    Si    K   Ca Ba Fe Type
1 1.52101 13.64 4.49 1.10 71.78 0.06 8.75  0  0    1
2 1.51761 13.89 3.60 1.36 72.73 0.48 7.83  0  0    1
3 1.51618 13.53 3.55 1.54 72.99 0.39 7.78  0  0    1
4 1.51766 13.21 3.69 1.29 72.61 0.57 8.22  0  0    1
5 1.51742 13.27 3.62 1.24 73.08 0.55 8.07  0  0    1
\end{Soutput}
\end{Schunk}

  \par Em sequência, iremos utilizar a normalização para que todas variáveis possam ter o mesmo peso no modelo, com a seguinte equação:
\begin{equation}
x'_{i} = \frac{x_i - min(x)}{max(x) - min(x)}
\end{equation}
  \par Neste problema as classes estão desbalanceadas como pode ser visto a seguir, onde cada coluna representa o \textit{label} e a linha representa a quantidade de dados de cada classe:
\begin{Schunk}
\begin{Soutput}
y
 1  2  3  5  6  7 
70 76 17 13  9 29 
\end{Soutput}
\end{Schunk}

  \par Por fim iremos separar 70\%  do conjunto para treino e 30\% para teste. Como neste exercício as classes estão desbalanceadas iremos forçar que todas tenham a mesma proporção do conjunto original. Para isso construímos a seguinte função:
\begin{Schunk}
\begin{Sinput}
> sample_proportion <- function(x, y){
+   labels <- names(table(y))
+   x_train <- matrix(ncol = ncol(x))
+   y_train <- c()
+   x_test <- matrix(ncol = ncol(x))
+   y_test <- c()
+   for(i in labels){
+     aux <- strtoi(i)
+     y_aux <- y[y==aux]
+     x_aux <- x[y==aux, ]
+     index <- sample(1:nrow(x_aux), length(1:nrow(x_aux)))
+     x_aux <- x_aux[index,1:ncol(x_aux)]
+     y_aux <- y_aux[index]
+     training_sample_number = round(nrow(x_aux)*0.7)+1
+     x_aux_train <- x_aux[1:training_sample_number,]
+     y_aux_train <- y_aux[1:training_sample_number]
+     x_aux_test <- x_aux[(training_sample_number+1):nrow(x_aux), ]
+     y_aux_test <- y_aux[(training_sample_number+1):nrow(x_aux)]
+     y_train <- c(y_train, y_aux_train)
+     x_train <- rbind(x_train, x_aux_train)
+     y_test <- c(y_test, y_aux_test)
+     x_test <- rbind(x_test, x_aux_test)
+   }
+   x_train <- x_train[2:nrow(x_train), 1:ncol(x_train)]
+   index <- sample(1:nrow(x_train), length(1:nrow(x_train)))
+   x_train <- x_train[index,1:ncol(x_train)]
+   y_train <- y_train[index]
+   x_test <- x_test[2:nrow(x_test), 1:ncol(x_test)]
+   index <- sample(1:nrow(x_test), length(1:nrow(x_test)))
+   x_test <- x_test[index,1:ncol(x_test)]
+   y_test <- y_test[index]
+   return(list(x_train,
+               y_train,
+               x_test,
+               y_test))
+ }
\end{Sinput}
\end{Schunk}


\section{Resultados}
  
  \par Neste experimento iremos para cada conjunto de parâmetros "kpar" e "C" faremos ele 10 vezes e por fim analisaremos a média e o desvio padrão.
  
\begin{Schunk}
\begin{Sinput}
> set.seed(1)
> accuracy_final <- matrix(nrow=10, ncol=9)
> for(i in 1:10){
+   samples <- sample_proportion(x, y)
+   x_train <- samples[[1]]
+   y_train <- samples[[2]]
+   x_test <- samples[[3]]
+   y_test <- samples[[4]]
+   r <- c(0.025, 0.025, 0.025, 0.015, 0.015, 0.015, 0.01, 0.01, 0.01)
+   C_par <- c(700, 500, 900, 800, 900, 930, 500, 600, 800)
+   results <- training_svm(x_train, y_train, x_test, y_test, r, C_par)
+   svm <- results[[1]]
+   accuracy_list <- results[[2]]
+   accuracy_final[i,] <- as.matrix(accuracy_list, nrow=1)
+ }
\end{Sinput}
\end{Schunk}

  \par Nas tabelas a seguir estão representadas as médias e os desvios padrão para cada experimento.
\begin{Schunk}
\begin{Soutput}
  colMeans.accuracy_final.
1                0.7474576
2                0.7440678
3                0.7440678
4                0.7406780
5                0.7423729
6                0.7406780
7                0.7440678
8                0.7406780
9                0.7406780
\end{Soutput}
\begin{Soutput}
          sd
1 0.04889536
2 0.05018400
3 0.04689606
4 0.06242874
5 0.06381931
6 0.06493489
7 0.05841382
8 0.05819483
9 0.06394422
\end{Soutput}
\end{Schunk}


\section{Discussão}

  \par Neste exercício os parâmetros “kpar” e “C” foram escolhidos novamente de forma aleatória, e foram ajustados conforme a acurácia dos resultados. Isso mostrou-se extremamente ineficiente. Para melhorar o desempenho do algoritmo sugiro duas hipóteses: (1) definir os parâmetros, pelo menos os primeiros, com alguma expressão envolvendo a dimensionalidade do problema; (2) rodar o algoritmo em paralelo com um conjunto de valores, e para cada conjunto escolher o melhor e criar um novo conjunto de acordo com uma distribuição de probabilidade. No final, comparar entre cada resultado obtido e escolher o melhor. O único problema do passo 2 pode ser o custo computacional para construí-lo.
  
   \par O melhor resultado obtido foi para "kpar" igual a 0.025 e "C" igual a 700. Nota-se que este valor ($\approx$ 75\%) é muito menor do que o obtido pelo problema anterior (100 \%). Esse problema possui 9 dimensões e 6 classes, o que aumenta consideravelmente a complexidade do problema.
  
  \par Logo, com o experimento foi possível compreender melhor o comportamento do algoritmo SVM, além de que conseguimos construir e validar um classificador SVM. É possível afirmar que a prática foi bem executada e o classificador construído apresentou um bom razoável. 
%%%%%%%%%%%%%%%%%%%%%%%%%%%%%%%%%%%%%%%%%%%%%%%%%%%%%%%%%%%%%%%%%%%%%%%%%%%%%%%%%%%%%%%%%

%%%%%%%%%%%%%%%%%%%%%%%%%%%%%%%%%%%%%%%%%%%%%%%%%%%%%%%%%%%%%%%%%%%%%%%%%%%%%%%%%%%%%%%%%



\end{document}
