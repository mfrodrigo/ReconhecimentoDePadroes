\documentclass[12pt]{article}
\usepackage[utf8]{inputenc}
%\usepackage[portuguese]{babel}
\usepackage{amsmath,amsfonts,amssymb}
\usepackage{graphicx}
\usepackage{makeidx}
\usepackage{graphicx}
\usepackage{lmodern}
\usepackage{multicol}
\usepackage{booktabs}
\usepackage{fancyhdr}
\usepackage{hyperref}
\usepackage[usenames]{color}


\usepackage{Sweave}
\begin{document}
\Sconcordance{concordance:perceptron_simples.tex:perceptron_simples.Rnw:%
1 15 1 1 0 60 1 1 83 85 0 1 2 1 9 11 0 1 2 38 1 1 3 2 0 1 3 1 0 1 3 1 0 %
1 3 1 0 1 1 1 3 1 0 1 1 1 3 1 0 1 1 3 0 1 2 4 1 1 2 1 0 1 30 32 0 1 2 3 %
1 1 2 1 0 1 1 5 0 1 1 5 0 1 1 5 0 1 1 6 0 1 2 14 1}

\pagestyle{fancy}
\fancyhf{}
\renewcommand{\headrulewidth}{0.4pt}
\fancyfoot[C]{\thepage}
\renewcommand{\footrulewidth}{0.4pt}
\fancyfoot[C]{\thepage}
\title{\LARGE \bf
 Exercício 1 - Aplicação perceptron simples}
\author{ Rodrigo Machado Fonseca - 2017002253}
\thispagestyle{fancy}
\fancyhead[C]{Introdução ao Reconhecimento de Padrões - UFMG \\ Belo Horizonte - \today}
\maketitle
\thispagestyle{fancy}

%%%%%%%%%%%%%%%%%%%%%%%%%%%%%%%%%%%%%%%%%%%%%%%%%%%%%%%%%%%%%%%%%%%%%%%%%%%%%%%%%%%%%%%%%
\section{Introdução}
  \par Neste trabalho nós utilizaremos o algoritmo Perceptron Simples para classificar se um tumor é benigno ou maligno de acordo com a base de dados \textit{Breast Cancer}$^{\ref{BreastCancer}}$.

\section{Perceptron Simples}

  \par A rede perceptron classifica baseado na função limiar que consiste na seguinte função:

\begin{equation}
\left \{
  \begin{array}{cc}
  w1*x1 + w2*x2 \geq \theta, & 1 \\
  w1*x1 + w2*x2  \theta, & 0 \\
  \end{array}
  \right.
  \label{eq:limiar1}
\end{equation}
  
\begin{equation*}
  \left \{
    \begin{array}{cc}
    w1,w2 & pesos \\
    x1,x2 & coordenadas \ \  de \ \ x \\
    \theta & limiar \ \  de \ \  separacao \\
    \end{array}
    \right.
\end{equation*}

 
  \par Os parâmetros que estamos interessados em descobrir são os pesos e o valor de $\theta$. Mas note que da forma que a equação $\ref{eq:limiar1}$ está disposta fica impossível utilizar o percepetron, pois após a multiplicação da entrada pelos pesos iremos comparar com o limiar, mas note que também não sabemos o limiar. Portanto, iremos fazer um pequeno ajuste na equação. 

\begin{equation}
    \left \{
      \begin{array}{cc}
      w1*x1 + w2*x2 - \theta \geq 0 , & 1 \\
      w1*x1 + w2*x2 - \theta < 0, & 0 \\
      \end{array}
      \right.
      \label{eq:limiar2}
\end{equation}

 \par Note que esta pequena alteração nos permite treinar a função de modo a descobrir os pesos e o limiar. Basta colocarmos uma entrada padrão -1 que será multiplicada por um valor aleatório de $\theta$, e com isso o valor de comparação da função degrau será centrado no zero. Portanto, o $\theta$ será atualizado assim como os pesos e no final a nossa função irá retornar $\theta, w1, w2$, respectivamente. 
      
  \par Com isso, agora podemos implementar a função $train\_perceptron$ e a função $calculate\_perceptron\_output$. Devo ressaltar que utilizaremos um número de peso arbitrário de modo que o algoritmo torne-se geral e possa ser aplicado independentemente da dimensão da entrada.  

\begin{Schunk}
\begin{Sinput}
> train_perceptron <- function(xin,yd,eta,tol,maxepocas,par){
+   
+   ## Definition of  functions parameters 
+   
+   #yd: output
+   #xin: input 
+   #eta: learning rate
+   #tol: maximum error tolerance
+   #maxepoch -> maximum number of iterations
+   #par: par=1, if p = 1, enter the term polarization
+   
+   ## Dimensions of input matrix
+   n_row <-dim(xin)[1]
+   n_col <-dim(xin)[2] 
+   
+   ## term polarization and weights 
+   
+   if (par==1){
+     
+     # initialize weight matrix
+     wt<-as.matrix(runif(n_col+1)-0.5) 
+     xin<-cbind(-1,xin)
+   }
+   else{
+     
+     # initialize weight matrix
+     wt<-as.matrix(runif(n_col)-0.5)
+   }
+   
+   ## control variables 
+   
+   # vector error per epoch
+   evec <- matrix(ncol = 1, nrow = 1)
+   
+   # variblas comparative  
+   nepoch <-1
+   error_epoch <-tol+1
+   
+   
+   while ((nepoch < maxepocas) && (error_epoch > tol))
+   {
+     # sum square error
+     ei2<-0
+     
+     # random string
+     xseq<-sample(n_row, replace = FALSE)
+     
+     for (i in 1:n_row)
+     {
+       
+       # random sequence
+       irand<-xseq[i]
+       
+       # output 
+       yhati<- as.double((xin[irand, ] %*% wt) >= 0)
+       # error 
+       ei<-yd[irand]-yhati
+       
+       # delta 
+       dw<-eta * ei * xin[irand,] 
+       
+       #Update wights 
+       wt<-wt + dw
+       
+       #erro acumulado
+       ei2<-ei2+ei*ei
+     }
+     
+     #mean square error (MSR)
+     evec <- rbind(evec, as.matrix(ei2/n_row, ncol = 1, nrow= 1))
+     
+     #save error per epoch
+     error_epoch <- ei2/n_row
+     
+     # Update number of epochs 
+     nepoch <-nepoch +1
+   }
+   
+   r_list <-list(wt, evec, nepoch)
+   return(r_list)
+   
+ } 
\end{Sinput}
\end{Schunk}

\begin{Schunk}
\begin{Sinput}
>   calculate_perceptron_output <- function(xvec,w,par){
+   if(par==1){
+       xvec<-cbind(-1,xvec)}
+   
+   u<-xvec %*% w
+   y<-1.0*(u>=0)
+   return ((as.matrix(y))) 
+ }
\end{Sinput}
\end{Schunk}

\section{Preparação dos Dados}

\par Na base \textit{Breast Cancer}, estão disponíveis nesta base 699 amostras, das quais 458 (65.5\%) correspondem a tumores benignos e 241 (34.5\%) a tumores malignos.
\begin{table}[!h]
        \centering
        \begin{tabular}{|c|c|c|}
            \hline
            Coluna & Atributo & Domínio \\
            \hline
             1. & Sample code number       &     id number \\
            \hline
            2. & Clump Thickness          &     1 - 10\\
             \hline
            3. & Uniformity of Cell Size    &   1 - 10\\
             \hline
            4. & Uniformity of Cell Shape    &  1 - 10\\
             \hline
            5. & Marginal Adhesion           &  1 - 10\\
             \hline
            6. & Single Epithelial Cell Size &  1 - 10\\
             \hline
            7. & Bare Nuclei                 &  1 - 10\\
             \hline
            8. & Bland Chromatin            &   1 - 10\\
             \hline
            9. & Normal Nucleoli            &   1 - 10\\
             \hline
           10. & Mitoses                   &    1 - 10\\
            \hline
           11. & Class:                   &     (2 for benign, 4 for malignant)\\
        	\hline 
    	\end{tabular} 
        \caption{Atributos da base de dados Breast Cancer}
        \label{tab:breastcancer}
\end{table}

\par Para utilizarmos esses dados vamos tratá-los de modo que torne-se possível a aplicação do Perceptron Simples. A priori, iremos remover os id's, pois eles não tem nenhuma correlação com o tipo de câncer. Em seguida, iremos excluir o valores NA's (não disponível). Por fim, iremos transformar benigno para 1 e maligno para 0, para ficar coerente a função degrau.

\begin{Schunk}
\begin{Sinput}
> # Carregar os dados
> data <- as.matrix(read.table("breast-cancer-wisconsin.data", sep = ','))
> # Remover coluna de ids
> data <- data[,-1]
> # Converter para matrix numerica
> data <- apply(data, c(1,2), as.numeric)
> # Remover linhas com NAs
> rows_to_remove <- apply(data, 1, function(x){any(is.na(x))})
> data <- data[!rows_to_remove,]
> # Obter amostras x e y
> x <- data[,1:9]
> y <- data[,10]
> # Rotular as amostras das Classes com o valor de 0 (maligno) e 1 (benigno)
> y[y == 4] <- 0
> y[y == 2] <- 1
\end{Sinput}
\end{Schunk}

\section{Treinamento}

  \par Para fazer o treinamento, iremos separar o conjunto de dados entre treinamento e teste. O conjunto de treinamento será utilizado para encontrar os pesos da nossa rede neural. O conjunto de teste será utilizado para analisar a capacidade do nosso classificador distinguir entre benigno e maligno. Esses dados devem ser escolhidos de forma aleatório de modo a não enviesar o estudo.
  
\begin{Schunk}
\begin{Sinput}
> library('RSNNS')
> train_routine <- function(x, y, eta, tol, maxepochs, par, number_repeat){
+   #y: Output variable array. 
+   #x: Input variable array.
+   #eta: Learning rate.
+   #tol: Maximum error tolerance.
+   #maxepochs -> Maximum number of iterations per epoch.
+   #par: par=1, if p = 1, enter the term polarization
+   #number_repeat: Number of analysis repetitions.
+   
+ accuracy <- matrix(nrow = number_repeat, ncol = 1)
+ for (i in 1:number_repeat)
+ {
+   # Select test and training samples
+   xy_all <- splitForTrainingAndTest(x,y,ratio = 0.3)
+   x_train <- xy_all$inputsTrain
+   y_train <- xy_all$targetsTrain
+   x_test <- xy_all$inputsTest
+   y_test <- xy_all$targetsTest
+   
+   # Training
+   retlist <- train_perceptron(x_train, y_train, eta, tol, maxepochs, par)
+   
+   # Testing
+   wt <- retlist[[1]]
+   y_hat <- calculate_perceptron_output(x_test, wt, par)
+   accuracy[i,] <- 1 - (t(y_test - y_hat) %*% (y_test - y_hat))/length(y_test)
+ }
+   
+ return(accuracy)
+ }
\end{Sinput}
\end{Schunk}

\section{Resultados}
  \par Agora que já possuíamos nossa rotina de trinamento podemos fazer o experimento. Iremos separa 70 \% para treino e 30 \% para teste. Iremos repetir o experimento 20 vezes e após isso calcular a média da acurácia e o seu desvio padrão.
  
\begin{Schunk}
\begin{Sinput}
> accuracy <- train_routine(x, y, 0.1, 0.0001, 1000, 1, 20)
> print("Acurácia Média")
\end{Sinput}
\begin{Soutput}
[1] "Acurácia Média"
\end{Soutput}
\begin{Sinput}
> mean(accuracy)
\end{Sinput}
\begin{Soutput}
[1] 0.9873171
\end{Soutput}
\begin{Sinput}
> print("Desvio Padrão")
\end{Sinput}
\begin{Soutput}
[1] "Desvio Padrão"
\end{Soutput}
\begin{Sinput}
> sd(accuracy)
\end{Sinput}
\begin{Soutput}
[1] 0.009146555
\end{Soutput}
\end{Schunk}

\section{Discussão}

  \par Com esse experimento, pode-se concluir que foi possível construir uma rede Perceptron Simples capaz de classificar corretamente as amostras mais de 95\% das vezes. Esse valor é bastante satisfatório, o que indica que o algoritmo de treinamento foi implementado corretamente e que uma rede Perceptron Simples é capaz de resolver o problema da base de dados \textit{Breast Cancer}. Além disso, ao escolhermos o conjunto de treino e teste  de forma aleatório, também atualizando os pesos de forma aleatória garantimos a replicabilidade deste experimento. 
%%%%%%%%%%%%%%%%%%%%%%%%%%%%%%%%%%%%%%%%%%%%%%%%%%%%%%%%%%%%%%%%%%%%%%%%%%%%%%%%%%%%%%%%%

%%%%%%%%%%%%%%%%%%%%%%%%%%%%%%%%%%%%%%%%%%%%%%%%%%%%%%%%%%%%%%%%%%%%%%%%%%%%%%%%%%%%%%%%%


\begin{thebibliography}{99}
		\bibitem{c1}\label{BreastCancer} \url{https://archive.ics.uci.edu/ml/datasets/Breast+Cancer+Wisconsin+\%28Original\%29}
\end{thebibliography}	


\end{document}
